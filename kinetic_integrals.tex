\documentclass[aps,prl,preprint,groupedaddress,10pt]{revtex4-2}
\usepackage{notations}
\usepackage{tikz}

\begin{document}
\section{General form of kinetic collision integrals}
In our model, we face three types of collision integrals, for each type of collision.
The aggregative integrals, the restitutive integrals and the fragmentative integrals.
We can solve the typical forms of each type of collision integrals in the most general form
\begin{equation}
    \begin{split}
        I_a^{k,l,m,p,q} = \int\dd{\bu}\dd{\bw}u^kw^l& e^{-Au^2-Bw^2+R\bu\vdot\bw}
        \Theta\pqty{\lambda_a-u}
        \int\dd{\bn}\Theta\pqty{-\bu\vdot\bn}\abs{\bu\vdot\bn}^m\pqty{\bu\vdot\bn}^p
        \pqty{\bw\vdot\bn}^q,\\
        I_r^{k,l,m,p,q} = \int\dd{\bu}\dd{\bw}u^kw^l& e^{-Au^2-Bw^2+R\bu\vdot\bw}
        \Theta\pqty{u-\lambda_a}
        \int\dd{\bn}\Theta\pqty{-\bu\vdot\bn}\abs{\bu\vdot\bn}^m\pqty{\bu\vdot\bn}^p
        \pqty{\bw\vdot\bn}^q\Theta\bqty{\lambda_f^2-\pqty{\bu\vdot\bn}^2},\\
        I_f^{k,l,m,p,q} = \int\dd{\bu}\dd{\bw}u^kw^l& e^{-Au^2-Bw^2+R\bu\vdot\bw}
        \int\dd{\bn}\Theta\pqty{-\bu\vdot\bn}\abs{\bu\vdot\bn}^m\pqty{\bu\vdot\bn}^p
        \pqty{\bw\vdot\bn}^q\Theta\bqty{\pqty{\bu\vdot\bn}^2-\lambda_f^2},
    \end{split}
\end{equation}
where $k,l,m,p,q,s$ are integers, and $q=\Bqty{0,1}$. The difference in each type of
integrals is in the domains of the vector $\bu$. In the aggregative case, the values of
$u$ have to be less than a certain threshold $\lambda_a$, in the restitutive case, the
values of $u$ have to be larger than $\lambda_a$, but restricted by the parameter
$\lambda_f$ from above. Finally, in the fragmentative case, the values of $u$ are restricted
by the parameter $\lambda_f$ from below.

\subsection{Angular integrals}
We start by first solving the inner integrals over $\bn$. By its physical meaning, we can call
them angular integrals. Note, that $q$ can be either $0$ or $1$, meaning that the
corresponding term either do exist or is absent
\begin{equation}
    \begin{split}
        I_{\bn,a}^{m,p,q}\pqty{\bu,\bw}&=\int\dd{\bn}\Theta\pqty{-\bu\vdot\bn}
        \abs{\bu\vdot\bn}^m\pqty{\bu\vdot\bn}^p\pqty{\bw\vdot\bn}^q,\\
        I_{\bn,r}^{m,p,q}\pqty{\bu,\bw}&=\int\dd{\bn}\Theta\pqty{-\bu\vdot\bn}
        \abs{\bu\vdot\bn}^m\pqty{\bu\vdot\bn}^p\pqty{\bw\vdot\bn}^q
        \Theta\bqty{\lambda_f^2-\pqty{\bu\vdot\bn}^2},\\
        I_{\bn,f}^{m,p,q}\pqty{\bu,\bw}&=\int\dd{\bn}\Theta\pqty{-\bu\vdot\bn}
        \abs{\bu\vdot\bn}^m\pqty{\bu\vdot\bn}^p\pqty{\bw\vdot\bn}^q
        \Theta\bqty{\pqty{\bu\vdot\bn}^2-\lambda_f^2}.\\
    \end{split}
\end{equation}
If $q=0$, then the angular integral is a function of only the vector $\bu$, otherwise it is
a function of both vectors $\bu$ and $\bw$.

Let us first solve the aggregative angular integrals
\subsubsection{Aggregative angular integrals}
We start with a simpler case when $q=0$ and the angular integral is a function of only $\bu$
\begin{equation}
    I_{\bn,a}^{m,p,0}(\bu)=\int\dd{\bn}\Theta\pqty{-\bu\vdot\bn}
    \abs{\bu\vdot\bn}^m\pqty{\bu\vdot\bn}^p.
\end{equation}
To solve this integral, we fix the vector $\bu$, and denote by $\theta$ the angle between
$\bu$ and $\bn$. In the spherical coordinates we have
$\dd{\bn}=\sin\theta\dd{\theta}\dd{\varphi}$, and the integral can be written as
\begin{equation}
    \begin{split}
        I_{\bn,a}^{m,p,0}(\bu)&=2\pi u^{m+p}\int_0^{\pi}\dd{\theta}\sin\theta
        \Theta\pqty{-\cos\theta}\abs{\cos\theta}^m\pqty{\cos\theta}^p=\\
        &=2\pi u^{m+p}\int_{\pi/2}^{\pi}\dd{\theta}\sin\theta
        \abs{\cos\theta}^m\pqty{\cos\theta}^p=\\
        &=-2\pi u^{m+p}\int_{\pi/2}^{\pi}\dd{\pqty{\cos\theta}}
        \abs{\cos\theta}^m\pqty{\cos\theta}^p,
    \end{split}
\end{equation}
where we have integrated out over $\varphi$ to give us the $2\pi$ factor. Now, substituting
$\cos\theta=z$, we write
\begin{equation}
    I_{\bn,a}^{m,p,0}(\bu)=2\pi u^{m+p}\int_{-1}^{0}\dd{z}\abs{z}^m z^p.
\end{equation}
Since in the integration domain $z$ is always negative, we now that $z^p<0$ for odd values of
$p$, and $z^p>0$ for even values of $p$, hence we can write $z^p=(-1)^p\abs{z}^p$, and
\begin{equation}
    I_{\bn,a}^{m,p,0}(\bu)=2\pi u^{m+p}\cdot(-1)^p\int_{-1}^{0}\dd{z}\abs{z}^{m+p}=
    -(-1)^p\cdot 2\pi u^{m+p}\int_{\pi/2}^{\pi}\dd{\pqty{\cos\theta}}\abs{\cos\theta}^{m+p}.
\end{equation}
We can see that for both odd and even values of $m+p$, the integral gives the same result,
and finally we have
\begin{equation}
    I_{\bn,a}^{m,p,0}(\bu)=(-1)^p\cdot\frac{2\pi u^{m+p}}{m+p+1}.
\end{equation}

The case with $q=1$ is trickier, since we have two arbitrary angles $\angle(\bn,\bu)$ and
$\angle(\bn,\bw)$. However, we can write it as a dot product of $\bw$ and another vector
$\bF$ as
\begin{equation}
    \begin{split}
        I_{\bn,a}^{m,p,1}(\bu,\bw)&=\int\dd{\bn}\Theta\pqty{-\bu\vdot\bn}
        \abs{\bu\vdot\bn}^m\pqty{\bu\vdot\bn}^p\pqty{\bw\vdot\bn}=\\
        &=\bw\vdot\int\dd{\bn}\Theta\pqty{-\bu\vdot\bn}
        \abs{\bu\vdot\bn}^m\pqty{\bu\vdot\bn}^p\bn=\bw\vdot\bF,
    \end{split}
\end{equation}
where the vector $\bF$ is constructed by vectors $\bn$ and $\bu$
\begin{equation}
    \bF = \int\dd{\bn}\Theta\pqty{-\bu\vdot\bn}\abs{\bu\vdot\bn}^m\pqty{\bu\vdot\bn}^p\bn.
\end{equation}
Since it is being integrated
over $\bn$, it cannot depend on $\bn$. This means that it can be oriented only along the
vector $\bu$, or $\bF=f\bu$. Now we can write
\begin{equation}
    \begin{split}
        u^2f = \bu\vdot\bF &= \bu\vdot
        \int\dd{\bn}\Theta\pqty{-\bu\vdot\bn}\abs{\bu\vdot\bn}^m\pqty{\bu\vdot\bn}^p\bn=\\
        &=\int\dd{\bn}\Theta\pqty{-\bu\vdot\bn}\abs{\bu\vdot\bn}^m\pqty{\bu\vdot\bn}^{p+1}=
        I_{\bn,a}^{m,p+1,0}(\bu),
    \end{split}
\end{equation}
or
\begin{equation}
    I_{\bn,a}^{m,p,1}(\bu,\bw)=f\bw\vdot\bu=\frac{\bw\vdot\bu}{u^2}
    \cdot I_{\bn,a}^{m,p+1,0}(\bu),
\end{equation}
which gives us the value of the integral
\begin{equation}
    I_{\bn,a}^{m,p,1}(\bu,\bw)=(-1)^{p+1}\cdot\frac{2\pi u^{m+p-1}}{m+p+2}\pqty{\bw\vdot\bu}.
\end{equation}
Now, we can combine both cases of $q=0$ and $q=1$, and write
\begin{equation}
    I_{\bn,a}^{m,p,q}(\bu,\bw)=\int\dd{\bn}\Theta\pqty{-\bu\vdot\bn}
    \abs{\bu\vdot\bn}^m\pqty{\bu\vdot\bn}^p\pqty{\bw\vdot\bn}^q=
    (-1)^{p+q}\cdot\frac{2\pi u^{m+p-q}}{m+p+q+1}
    \cdot\pqty{\bw\vdot\bu}^{q},\quad q=\Bqty{0,1}.
\end{equation}

\subsubsection{Restitutive angular integrals}
These type of integrals have a domain restriction terms given by the parameter $\lambda_f$.
We can start with a simpler case when $q=0$, and write
\begin{equation}
    \begin{split}
        I_{\bn,r}^{m,p,0}(\bu)&=\int\dd{\bn}\Theta\pqty{-\bu\vdot\bn}
        \abs{\bu\vdot\bn}^m\pqty{\bu\vdot\bn}^p
        \Theta\bqty{\lambda_f^2-\pqty{\bu\vdot\bn}^2}=\\
        &=(-1)^p\int\dd{\bn}\Theta\pqty{-\bu\vdot\bn}
        \abs{\bu\vdot\bn}^{m+p}
        \Theta\bqty{\lambda_f^2-\pqty{\bu\vdot\bn}^2}.
    \end{split}
\end{equation}
Again, switching to spherical coordinates, and denoting the angle $\angle(\bn,\bu)$ by
$\theta$, we write
\begin{equation}
    \begin{split}
        I_{\bn,r}^{m,p,0}(\bu)&=(-1)^p\cdot 2\pi u^{m+p}\int_0^{\pi}\dd{\theta}\sin\theta
        \Theta\pqty{-\cos\theta}\abs{\cos\theta}^{m+p}
        \Theta\bqty{\lambda_f^2-\pqty{u\cos\theta}^2}=\\
        &=(-1)^p\cdot 2\pi u^{m+p}\int_{\pi/2}^{\pi}\dd{\theta}\sin\theta
        \abs{\cos\theta}^{m+p}\Theta\bqty{\frac{\lambda_f^2}{u^2}-\cos^2\theta}.
    \end{split}
\end{equation}
The domain restriction implies
\begin{equation}
    \frac{\lambda_f}{u}\geqslant\abs{\cos\theta}.
\end{equation}
This constraint restricts two variable, both $\theta$ and $u$, although we do not perform
integration over $u$ at this moment. Since the variable $u$ changes from $0$ to $\infty$,
the restriction can be split into two cases, (i) when $u\leqslant\lambda_f$, (ii) when
$u>\lambda_f$. In the first case, when $u\leqslant\lambda_f$, the restriction holds true
for any values of $\theta\in\bqty{\pi/2,\pi}$, e.g. no constraint in the angle $\theta$.
In the second case, when $u>\lambda_f$, the restriction holds true only within a certain
range of values of $\theta$, namely $\theta\in\bqty{\pi/2,\pi-\arccos\pqty{\lambda_f/u}}$.
Now, we can rewrite the domain restriction term as
\begin{equation}
    \Theta\bqty{\frac{\lambda_f^2}{u^2}-\cos^2\theta}=
    \Theta\pqty{\lambda_f-u}+\Theta\pqty{u-\lambda_f}
    \Theta\bqty{\pi-\arccos\pqty{\frac{\lambda_f}{u}}-\theta}.
\end{equation}
Using this form of the restriction allows us to solve the restitutive angular integrals
\begin{equation}
    \begin{split}
        I_{\bn,r}^{m,p,0}(\bu)=&-(-1)^p\cdot 2\pi u^{m+p}\Theta\pqty{\lambda_f-u}
        \int_{\pi/2}^{\pi}\dd{\pqty{\cos\theta}}\abs{\cos\theta}^{m+p}-\\
        &-(-1)^p\cdot 2\pi u^{m+p}\Theta\pqty{u-\lambda_f}
        \int_{\pi/2}^{\pi-\arccos(\lambda_f/u)}\dd{\pqty{\cos\theta}}\abs{\cos\theta}^{m+p}.
    \end{split}
\end{equation}
The first integral is already solved for the aggregative angular case, and in the
second integral we substitute $z=\cos\theta$, and write
\begin{equation}
    \begin{split}
        I_{\bn,r}^{m,p,0}(\bu)&=\Theta\pqty{\lambda_f-u}\cdot I_{\bn,a}^{m,p,0}(\bu)-
        (-1)^p\cdot 2\pi u^{m+p}\Theta\pqty{u-\lambda_f}
        \int_{0}^{-\lambda_f/u}\dd{z}\abs{z}^{m+p}=\\
        &=(-1)^p\cdot\Theta\pqty{\lambda_f-u}\cdot
        \frac{2\pi u^{m+p}}{m+p+1}+
        (-1)^p\cdot \Theta\pqty{u-\lambda_f}\cdot\frac{2\pi u^{m+p}}{m+p+1}
        \pqty{\frac{\lambda_f}{u}}^{m+p+1}=\\
        &=(-1)^p\cdot\frac{2\pi u^{m+p}}{m+p+1}
        \bqty{\Theta\pqty{\lambda_f-u}+\Theta\pqty{u-\lambda_f}
            \pqty{\frac{\lambda_f}{u}}^{m+p+1}},
    \end{split}
\end{equation}
or
\begin{equation}
    I_{\bn,r}^{m,p,0}(\bu)=I_{\bn,a}^{m,p,0}(\bu)\cdot
    \bqty{\Theta\pqty{\lambda_f-u}+\Theta\pqty{u-\lambda_f}\pqty{\frac{\lambda_f}{u}}^{m+p+1}}.
\end{equation}

For the case $q=1$, we can perform the same procedure as before, and write
\begin{equation}
    I_{\bn,r}^{m,p,1}\pqty{\bu,\bw}=\bw\vdot\int\dd{\bn}\Theta\pqty{-\bu\vdot\bn}
    \abs{\bu\vdot\bn}^m\pqty{\bu\vdot\bn}^p
    \Theta\bqty{\lambda_f^2-\pqty{\bu\vdot\bn}^2}\bn=\bw\vdot\bF,
\end{equation}
where
\begin{equation}
    \bF = \int\dd{\bn}\Theta\pqty{-\bu\vdot\bn}
    \abs{\bu\vdot\bn}^m\pqty{\bu\vdot\bn}^p
    \Theta\bqty{\lambda_f^2-\pqty{\bu\vdot\bn}^2}\bn.
\end{equation}
Again, we see that $\bF$ vector cannot depend on $\bn$, and depends only on the vector
$\bu$. This implies that $\bF=f\bu$, or
\begin{equation}
    f=\frac{\bF\vdot\bu}{u^2}=\frac{1}{u^2}\int\dd{\bn}\Theta\pqty{-\bu\vdot\bn}
    \abs{\bu\vdot\bn}^m\pqty{\bu\vdot\bn}^{p+1}
    \Theta\bqty{\lambda_f^2-\pqty{\bu\vdot\bn}^2}=
    u^{-2}\cdot I_{\bn,r}^{m,p+1,0}(\bu).
\end{equation}
Since,
\begin{equation}
    I_{\bn,r}^{m,p,1}(\bu,\bw)=\bw\vdot\bF=\pqty{\bw\vdot\bu}f=
    \frac{\bw\vdot\bu}{u^2}\cdot I_{\bn,r}^{m,p+1,0}(\bu),
\end{equation}
or writing explicitly, we have
\begin{equation}
    I_{\bn,r}^{m,p,1}(\bu,\bw)=(-1)^{p+1}\cdot\frac{2\pi u^{m+p-1}}{m+p+2}
    \bqty{\Theta\pqty{\lambda_f-u}+\Theta\pqty{u-\lambda_f}
        \pqty{\frac{\lambda_f}{u}}^{m+p+2}}\pqty{\bw\vdot\bu}.
\end{equation}
By combining both cases $q=0$ and $q=1$, we write the final solution of the
restitutive angular integrals as
\begin{equation}
    I_{\bn,r}^{m,p,q}(\bu,\bw)=(-1)^{p+q}\cdot\frac{2\pi u^{m+p-q}}{m+p+q+1}
    \bqty{\Theta\pqty{\lambda_f-u}+\Theta\pqty{u-\lambda_f}
        \pqty{\frac{\lambda_f}{u}}^{m+p+q+1}}\pqty{\bw\vdot\bu}^q.
\end{equation}

\subsubsection{Fragmentative angular integrals}
The last type of angular integrals is the fragmentative type, which is very
similar to the restitutive angular case.



\end{document}