\documentclass[aps,prl,preprint,groupedaddress,10pt]{revtex4-2}
\usepackage{notations}
\usepackage{tikz}

\begin{document}
\section{General form of kinetic collision integrals}
In our model, we face three types of collision integrals, for each type of collision.
The aggregative integrals, the restitutive integrals and the fragmentative integrals.
We can solve the typical forms of each type of collision integrals in the most general form
\begin{equation}
    \begin{split}
        I_a^{k,l,m,p,q} &= \int\dd{\bu}\dd{\bw}u^kw^{l}
        e^{-Aw^2-Bu^2+R\bw\vdot\bu}\Theta\pqty{\lambda_a-u}
        \int\dd{\bn}\Theta\pqty{-\bu\vdot\bn}\abs{\bu\vdot\bn}^m
        \pqty{\bu\vdot\bn}^p\pqty{\bw\vdot\bn}^q,\\
        I_r^{k,l,m,p,q} &= \int\dd{\bu}\dd{\bw}u^kw^{l}
        e^{-Aw^2-Bu^2+R\bw\vdot\bu}\Theta\pqty{u-\lambda_a}
        \int\dd{\bn}\Theta\pqty{-\bu\vdot\bn}\abs{\bu\vdot\bn}^m
        \pqty{\bu\vdot\bn}^p\pqty{\bw\vdot\bn}^q
        \Theta\bqty{\lambda_f^2-\pqty{\bu\vdot\bn}^2},\\
        I_f^{k,l,m,p,q} &= \int\dd{\bu}\dd{\bw}u^kw^{l}
        e^{-Aw^2-Bu^2+R\bw\vdot\bu}
        \int\dd{\bn}\Theta\pqty{-\bu\vdot\bn}\abs{\bu\vdot\bn}^m
        \pqty{\bu\vdot\bn}^p\pqty{\bw\vdot\bn}^q
        \Theta\bqty{\pqty{\bu\vdot\bn}^2-\lambda_f^2},\\
    \end{split}
\end{equation}
where $k,l,m,p,q$ are integers, and $q=\Bqty{0,1}$. The difference in each type of
integrals is in the domains of the vector $\bu$. In the aggregative case, the values of
$u$ have to be less than a certain threshold $\lambda_a$, in the restitutive case, the
values of $u$ have to be larger than $\lambda_a$, but restricted by the parameter
$\lambda_f$ from above. Finally, in the fragmentative case, the values of $u$ are restricted
by the parameter $\lambda_f$ from below.

\subsection{Angular integrals}
We start by first solving the inner integrals over $\bn$. By its physical meaning, we can call
them angular integrals. Note, that $q$ can be either $0$ or $1$, meaning that the
corresponding term either do exist or is absent
\begin{equation}
    \begin{split}
        I_{\bn,a}^{m,p,q}\pqty{\bu,\bw}&=\int\dd{\bn}\Theta\pqty{-\bu\vdot\bn}
        \abs{\bu\vdot\bn}^m\pqty{\bu\vdot\bn}^p\pqty{\bw\vdot\bn}^q,\\
        I_{\bn,r}^{m,p,q}\pqty{\bu,\bw}&=\int\dd{\bn}\Theta\pqty{-\bu\vdot\bn}
        \abs{\bu\vdot\bn}^m\pqty{\bu\vdot\bn}^p\pqty{\bw\vdot\bn}^q
        \Theta\bqty{\lambda_f^2-\pqty{\bu\vdot\bn}^2},\\
        I_{\bn,f}^{m,p,q}\pqty{\bu,\bw}&=\int\dd{\bn}\Theta\pqty{-\bu\vdot\bn}
        \abs{\bu\vdot\bn}^m\pqty{\bu\vdot\bn}^p\pqty{\bw\vdot\bn}^q
        \Theta\bqty{\pqty{\bu\vdot\bn}^2-\lambda_f^2}.\\
    \end{split}
\end{equation}
If $q=0$, then the angular integral is a function of only the vector $\bu$, otherwise it is
a function of both vectors $\bu$ and $\bw$.

Let us first solve the aggregative angular integrals
\subsubsection{Aggregative angular integrals}
We start with a simpler case when $q=0$ and the angular integral is a function of only $\bu$
\begin{equation}
    I_{\bn,a}^{m,p,0}(\bu)=\int\dd{\bn}\Theta\pqty{-\bu\vdot\bn}
    \abs{\bu\vdot\bn}^m\pqty{\bu\vdot\bn}^p.
\end{equation}
To solve this integral, we fix the vector $\bu$, and denote by $\theta$ the angle between
$\bu$ and $\bn$. In the spherical coordinates we have
$\dd{\bn}=\sin\theta\dd{\theta}\dd{\varphi}$, and the integral can be written as
\begin{equation}
    \begin{split}
        I_{\bn,a}^{m,p,0}(\bu)&=2\pi u^{m+p}\int_0^{\pi}\dd{\theta}\sin\theta
        \Theta\pqty{-\cos\theta}\abs{\cos\theta}^m\pqty{\cos\theta}^p=\\
        &=2\pi u^{m+p}\int_{\pi/2}^{\pi}\dd{\theta}\sin\theta
        \abs{\cos\theta}^m\pqty{\cos\theta}^p=\\
        &=-2\pi u^{m+p}\int_{\pi/2}^{\pi}\dd{\pqty{\cos\theta}}
        \abs{\cos\theta}^m\pqty{\cos\theta}^p,
    \end{split}
\end{equation}
where we have integrated out over $\varphi$ to give us the $2\pi$ factor. Now, substituting
$\cos\theta=z$, we write
\begin{equation}
    I_{\bn,a}^{m,p,0}(\bu)=2\pi u^{m+p}\int_{-1}^{0}\dd{z}\abs{z}^m z^p.
\end{equation}
Since in the integration domain $z$ is always negative, we now that $z^p<0$ for odd values of
$p$, and $z^p>0$ for even values of $p$, hence we can write $z^p=(-1)^p\abs{z}^p$, and
\begin{equation}
    I_{\bn,a}^{m,p,0}(\bu)=2\pi u^{m+p}\cdot(-1)^p\int_{-1}^{0}\dd{z}\abs{z}^{m+p}=
    -(-1)^p\cdot 2\pi u^{m+p}\int_{\pi/2}^{\pi}\dd{\pqty{\cos\theta}}\abs{\cos\theta}^{m+p}.
\end{equation}
We can see that for both odd and even values of $m+p$, the integral gives the same result,
and finally we have
\begin{equation}
    I_{\bn,a}^{m,p,0}(\bu)=(-1)^p\cdot\frac{2\pi u^{m+p}}{m+p+1}.
\end{equation}

The case with $q=1$ is trickier, since we have two arbitrary angles $\angle(\bn,\bu)$ and
$\angle(\bn,\bw)$. However, we can write it as a dot product of $\bw$ and another vector
$\bF$ as
\begin{equation}
    \begin{split}
        I_{\bn,a}^{m,p,1}(\bu,\bw)&=\int\dd{\bn}\Theta\pqty{-\bu\vdot\bn}
        \abs{\bu\vdot\bn}^m\pqty{\bu\vdot\bn}^p\pqty{\bw\vdot\bn}=\\
        &=\bw\vdot\int\dd{\bn}\Theta\pqty{-\bu\vdot\bn}
        \abs{\bu\vdot\bn}^m\pqty{\bu\vdot\bn}^p\bn=\bw\vdot\bF,
    \end{split}
\end{equation}
where the vector $\bF$ is constructed by vectors $\bn$ and $\bu$
\begin{equation}
    \bF = \int\dd{\bn}\Theta\pqty{-\bu\vdot\bn}\abs{\bu\vdot\bn}^m\pqty{\bu\vdot\bn}^p\bn.
\end{equation}
Since it is being integrated
over $\bn$, it cannot depend on $\bn$. This means that it can be oriented only along the
vector $\bu$, or $\bF=f\bu$. Now we can write
\begin{equation}
    \begin{split}
        u^2f = \bu\vdot\bF &= \bu\vdot
        \int\dd{\bn}\Theta\pqty{-\bu\vdot\bn}\abs{\bu\vdot\bn}^m\pqty{\bu\vdot\bn}^p\bn=\\
        &=\int\dd{\bn}\Theta\pqty{-\bu\vdot\bn}\abs{\bu\vdot\bn}^m\pqty{\bu\vdot\bn}^{p+1}=
        I_{\bn,a}^{m,p+1,0}(\bu),
    \end{split}
\end{equation}
or
\begin{equation}
    I_{\bn,a}^{m,p,1}(\bu,\bw)=f\bw\vdot\bu=\frac{\bw\vdot\bu}{u^2}
    \cdot I_{\bn,a}^{m,p+1,0}(\bu),
\end{equation}
which gives us the value of the integral
\begin{equation}
    I_{\bn,a}^{m,p,1}(\bu,\bw)=(-1)^{p+1}\cdot\frac{2\pi u^{m+p-1}}{m+p+2}\pqty{\bw\vdot\bu}.
\end{equation}
Now, we can combine both cases of $q=0$ and $q=1$, and write
\begin{equation}
    I_{\bn,a}^{m,p,q}(\bu,\bw)=\int\dd{\bn}\Theta\pqty{-\bu\vdot\bn}
    \abs{\bu\vdot\bn}^m\pqty{\bu\vdot\bn}^p\pqty{\bw\vdot\bn}^q=
    (-1)^{p+q}\cdot\frac{2\pi u^{m+p-q}}{m+p+q+1}
    \cdot\pqty{\bw\vdot\bu}^{q},\quad q=\Bqty{0,1}.
\end{equation}

\subsubsection{Restitutive angular integrals}
These type of integrals have a domain restriction terms given by the parameter $\lambda_f$.
We can start with a simpler case when $q=0$, and write
\begin{equation}
    \begin{split}
        I_{\bn,r}^{m,p,0}(\bu)&=\int\dd{\bn}\Theta\pqty{-\bu\vdot\bn}
        \abs{\bu\vdot\bn}^m\pqty{\bu\vdot\bn}^p
        \Theta\bqty{\lambda_f^2-\pqty{\bu\vdot\bn}^2}=\\
        &=(-1)^p\int\dd{\bn}\Theta\pqty{-\bu\vdot\bn}
        \abs{\bu\vdot\bn}^{m+p}
        \Theta\bqty{\lambda_f^2-\pqty{\bu\vdot\bn}^2}.
    \end{split}
\end{equation}
Again, switching to spherical coordinates, and denoting the angle $\angle(\bn,\bu)$ by
$\theta$, we write
\begin{equation}
    \begin{split}
        I_{\bn,r}^{m,p,0}(\bu)&=(-1)^p\cdot 2\pi u^{m+p}\int_0^{\pi}\dd{\theta}\sin\theta
        \Theta\pqty{-\cos\theta}\abs{\cos\theta}^{m+p}
        \Theta\bqty{\lambda_f^2-\pqty{u\cos\theta}^2}=\\
        &=(-1)^p\cdot 2\pi u^{m+p}\int_{\pi/2}^{\pi}\dd{\theta}\sin\theta
        \abs{\cos\theta}^{m+p}\Theta\bqty{\frac{\lambda_f^2}{u^2}-\cos^2\theta}.
    \end{split}
\end{equation}
The domain restriction implies
\begin{equation}
    \abs{\cos\theta}\leqslant\frac{\lambda_f}{u}.
\end{equation}
This constraint restricts two variable, both $\theta$ and $u$, although we do not perform
integration over $u$ at this moment. Since the variable $u$ changes from $0$ to $\infty$,
the restriction can be split into two cases, (i) when $u\leqslant\lambda_f$, (ii) when
$u>\lambda_f$. In the first case, when $u\leqslant\lambda_f$, the restriction holds true
for any values of $\theta\in\bqty{\pi/2,\pi}$, e.g. no constraint in the angle $\theta$.
In the second case, when $u>\lambda_f$, the restriction holds true only within a certain
range of values of $\theta$, namely $\theta\in\bqty{\pi/2,\pi-\arccos\pqty{\lambda_f/u}}$.
Now, we can rewrite the domain restriction term as
\begin{equation}
    \Theta\bqty{\frac{\lambda_f^2}{u^2}-\cos^2\theta}=
    \Theta\pqty{\lambda_f-u}+\Theta\pqty{u-\lambda_f}
    \Theta\bqty{\pi-\arccos\pqty{\frac{\lambda_f}{u}}-\theta}.
\end{equation}
Using this form of the restriction allows us to solve the restitutive angular integrals
\begin{equation}
    \begin{split}
        I_{\bn,r}^{m,p,0}(\bu)=&-(-1)^p\cdot 2\pi u^{m+p}\Theta\pqty{\lambda_f-u}
        \int_{\pi/2}^{\pi}\dd{\pqty{\cos\theta}}\abs{\cos\theta}^{m+p}-\\
        &-(-1)^p\cdot 2\pi u^{m+p}\Theta\pqty{u-\lambda_f}
        \int_{\pi/2}^{\pi-\arccos(\lambda_f/u)}\dd{\pqty{\cos\theta}}\abs{\cos\theta}^{m+p}.
    \end{split}
\end{equation}
The first integral is already solved for the aggregative angular case, and in the
second integral we substitute $z=\cos\theta$, and write
\begin{equation}
    \begin{split}
        I_{\bn,r}^{m,p,0}(\bu)&=\Theta\pqty{\lambda_f-u}\cdot I_{\bn,a}^{m,p,0}(\bu)-
        (-1)^p\cdot 2\pi u^{m+p}\Theta\pqty{u-\lambda_f}
        \int_{0}^{-\lambda_f/u}\dd{z}\abs{z}^{m+p}=\\
        &=(-1)^p\cdot\Theta\pqty{\lambda_f-u}\cdot
        \frac{2\pi u^{m+p}}{m+p+1}+
        (-1)^p\cdot \Theta\pqty{u-\lambda_f}\cdot\frac{2\pi u^{m+p}}{m+p+1}
        \pqty{\frac{\lambda_f}{u}}^{m+p+1}=\\
        &=(-1)^p\cdot\frac{2\pi u^{m+p}}{m+p+1}
        \bqty{\Theta\pqty{\lambda_f-u}+\Theta\pqty{u-\lambda_f}
            \pqty{\frac{\lambda_f}{u}}^{m+p+1}},
    \end{split}
\end{equation}
or
\begin{equation}
    I_{\bn,r}^{m,p,0}(\bu)=I_{\bn,a}^{m,p,0}(\bu)\cdot
    \bqty{\Theta\pqty{\lambda_f-u}+\Theta\pqty{u-\lambda_f}\pqty{\frac{\lambda_f}{u}}^{m+p+1}}.
\end{equation}

For the case $q=1$, we can perform the same procedure as before, and write
\begin{equation}
    I_{\bn,r}^{m,p,1}\pqty{\bu,\bw}=\bw\vdot\int\dd{\bn}\Theta\pqty{-\bu\vdot\bn}
    \abs{\bu\vdot\bn}^m\pqty{\bu\vdot\bn}^p
    \Theta\bqty{\lambda_f^2-\pqty{\bu\vdot\bn}^2}\bn=\bw\vdot\bF,
\end{equation}
where
\begin{equation}
    \bF = \int\dd{\bn}\Theta\pqty{-\bu\vdot\bn}
    \abs{\bu\vdot\bn}^m\pqty{\bu\vdot\bn}^p
    \Theta\bqty{\lambda_f^2-\pqty{\bu\vdot\bn}^2}\bn.
\end{equation}
Again, we see that $\bF$ vector cannot depend on $\bn$, and depends only on the vector
$\bu$. This implies that $\bF=f\bu$, or
\begin{equation}
    f=\frac{\bF\vdot\bu}{u^2}=\frac{1}{u^2}\int\dd{\bn}\Theta\pqty{-\bu\vdot\bn}
    \abs{\bu\vdot\bn}^m\pqty{\bu\vdot\bn}^{p+1}
    \Theta\bqty{\lambda_f^2-\pqty{\bu\vdot\bn}^2}=
    u^{-2}\cdot I_{\bn,r}^{m,p+1,0}(\bu).
\end{equation}
Since,
\begin{equation}
    I_{\bn,r}^{m,p,1}(\bu,\bw)=\bw\vdot\bF=\pqty{\bw\vdot\bu}f=
    \frac{\bw\vdot\bu}{u^2}\cdot I_{\bn,r}^{m,p+1,0}(\bu),
\end{equation}
or writing explicitly, we have
\begin{equation}
    I_{\bn,r}^{m,p,1}(\bu,\bw)=(-1)^{p+1}\cdot\frac{2\pi u^{m+p-1}}{m+p+2}
    \bqty{\Theta\pqty{\lambda_f-u}+\Theta\pqty{u-\lambda_f}
        \pqty{\frac{\lambda_f}{u}}^{m+p+2}}\pqty{\bw\vdot\bu}.
\end{equation}
By combining both cases $q=0$ and $q=1$, we write the final solution of the
restitutive angular integrals as
\begin{equation}
    I_{\bn,r}^{m,p,q}(\bu,\bw)=(-1)^{p+q}\cdot\frac{2\pi u^{m+p-q}}{m+p+q+1}
    \bqty{\Theta\pqty{\lambda_f-u}+\Theta\pqty{u-\lambda_f}
        \pqty{\frac{\lambda_f}{u}}^{m+p+q+1}}\pqty{\bw\vdot\bu}^q.
\end{equation}

\subsubsection{Fragmentative angular integrals}
The last type of angular integrals is the fragmentative type, which is very
similar to the restitutive angular case. Again, we start with the simpler case of
$q=0$
\begin{equation}
    I_{\bn,f}^{m,p,0}\pqty{\bu}=\int\dd{\bn}\Theta\pqty{-\bu\vdot\bn}
    \abs{\bu\vdot\bn}^m\pqty{\bu\vdot\bn}^p
    \Theta\bqty{\pqty{\bu\vdot\bn}^2-\lambda_f^2}.
\end{equation}
The difference of this type of angular integrals is in the inverse domain restriction
function. Switching into spherical coordinates, we write
\begin{equation}
    I_{\bn,f}^{m,p,0}\pqty{\bu}=-(-1)^p\cdot 2\pi u^{m+p}\int_{\pi/2}^{\pi}
    \dd{\pqty{\cos\theta}}\abs{\cos\theta}^{m+p}
    \Theta\bqty{\cos^2\theta-\frac{\lambda_f^2}{u^2}}.
\end{equation}
The domain restriction is now given as
\begin{equation}
    \abs{\cos\theta}\geqslant\frac{\lambda_f}{u}.
\end{equation}
This condition can be satisfied only for $u\geqslant\lambda_f$, and we can rewrite
the domain restriction as
\begin{equation}
    \Theta\bqty{\cos^2\theta-\frac{\lambda_f^2}{u^2}}=
    \Theta\pqty{u-\lambda_f}\Theta\bqty{\theta-\pi-\arccos\pqty{\frac{\lambda_f}{u}}},
\end{equation}
and our fragmentative angular integral becomes
\begin{equation}
    \begin{split}
        I_{\bn,f}^{m,p,0}\pqty{\bu}&=-(-1)^p\cdot 2\pi u^{m+p}\Theta\pqty{u-\lambda_f}
        \int_{\pi-\arccos(\lambda_f/u)}^{\pi}\dd{\pqty{\cos\theta}}\abs{\cos\theta}^{m+p}=\\
        &=-(-1)^p\cdot 2\pi u^{m+p}\Theta\pqty{u-\lambda_f}
        \int_{-\lambda_f/u}^{-1}\dd{z}\abs{z}^{m+p}=\\
        &=(-1)^p\cdot\frac{2\pi u^{m+p}}{m+p+1}\bqty{1-\pqty{\frac{\lambda_f}{u}}^{m+p+1}}
        \Theta\pqty{u-\lambda_f}.
    \end{split}
\end{equation}

The case with $q=1$ can be solved exactly as the previous cases, and we can immediately
write
\begin{equation}
    I_{\bn,f}^{m,p,1}(\bu,\bw)=\frac{\bw\vdot\bu}{u^2}\cdot I_{\bn,f}^{m,p+1,0}(\bu)=
    (-1)^{p+1}\cdot\frac{2\pi u^{m+p-1}}{m+p+2}\bqty{1-\pqty{\frac{\lambda_f}{u}}^{m+p+2}}
    \Theta\pqty{u-\lambda_f}\pqty{\bw\vdot\bu},
\end{equation}
and combining both cases $q=0$ and $q=1$, we have
\begin{equation}
    I_{\bn,f}^{m,p,q}(\bu,\bw)=(-1)^{p+q}\cdot\frac{2\pi u^{m+p-q}}{m+p+q+1}
    \bqty{1-\pqty{\frac{\lambda_f}{u}}^{m+p+q+1}}
    \Theta\pqty{u-\lambda_f}\pqty{\bw\vdot\bu}^q.
\end{equation}

\subsubsection{Final results of angular integrals}
Let us write the final results of solutions of the angular integrals for all three types
of collision integrals. First, the solution of the aggregative angular integrals is
\begin{equation}
    I_{\bn,a}^{m,p,q}(\bu,\bw)=\int\dd{\bn}\Theta\pqty{-\bu\vdot\bn}
    \abs{\bu\vdot\bn}^m\pqty{\bu\vdot\bn}^p\pqty{\bw\vdot\bn}^q=
    (-1)^{p+q}\cdot\frac{2\pi u^{m+p-q}}{m+p+q+1}
    \cdot\pqty{\bw\vdot\bu}^{q},\quad q=\Bqty{0,1}.
\end{equation}
The restitutive and fragmentative angular integrals are solved to give us
\begin{equation}
    \begin{split}
        I_{\bn,r}^{m,p,q}(\bu,\bw)&=I_{\bn,a}^{m,p,q}(\bu,\bw)\cdot
        \bqty{\Theta\pqty{\lambda_f-u}+\Theta\pqty{u-\lambda_f}
            \pqty{\frac{\lambda_f}{u}}^{m+p+q+1}},\\
        I_{\bn,f}^{m,p,q}(\bu,\bw)&=I_{\bn,a}^{m,p,q}(\bu,\bw)\cdot
        \bqty{1-\pqty{\frac{\lambda_f}{u}}^{m+p+q+1}}
        \Theta\pqty{u-\lambda_f}.
    \end{split}
\end{equation}

\subsection{Center of mass velocity integrals}
We refer to the integrals over the vector $\bw$ as the center of mass velocity
integrals. All three types of collision integrals contain similar forms of the
center of mass velocity integrals, and we can write a generic form of such integrals
as
\begin{equation}
    I_{\bw}^{l,q}(\bu)=\int\dd{\bw}w^{l}\pqty{\bw\vdot\bu}^q\exp\pqty{-Aw^2}
    \exp\pqty{R\bw\vdot\bu},\quad q=\Bqty{0,1}.
\end{equation}

Switching into spherical coordinates, and denoting by $\theta$ the angle between vectors
$\bw$ and $\bu$, we have $\dd{\bw}=w^2\sin\theta\dd{w}\dd{\theta}\dd{\varphi}$,
\begin{equation}
    I_{\bw}^{l,q}(\bu)=-2\pi u^q\int_0^{\infty}\dd{w}w^{l+q+2}\exp\pqty{-Aw^2}
    \int_0^{\pi}\dd{\pqty{\cos\theta}}\pqty{\cos\theta}^q\exp\pqty{Rwu\cdot\cos\theta},
    \quad q=\Bqty{0,1}.
\end{equation}
Again, we solve these integrals for two different cases of $q$, starting with the
simpler case

\subsubsection{The case with $q=0$}
In this case we write
\begin{equation}
    I_{\bw}^{l,0}(\bu)=-2\pi\int_0^{\infty}\dd{w}w^{l+2}\exp\pqty{-Aw^2}
    \int_0^{\pi}\dd{\pqty{\cos\theta}}\exp\pqty{Rwu\cdot\cos\theta}.
\end{equation}
The inner angular integral is solved to give us
\begin{equation}
    I^0_R(\bu,\bw)=-\int_0^{\pi}\dd{\pqty{\cos\theta}}\exp\pqty{Rwu\cdot\cos\theta}=
    \frac{1}{Rwu}\pqty{e^{Rwu}-e^{-Rwu}},
\end{equation}
and substituting into the center of mass velocity integral, we have
\begin{equation}
    I_{\bw}^{l,0}(\bu)=\frac{2\pi}{Ru}\int_0^{\infty}\dd{w}w^{l+1}\exp\pqty{-Aw^2}
    \pqty{e^{Ru\cdot w}-e^{-Ru\cdot w}}.
\end{equation}


\subsubsection{The case with $q=1$}
The case with $q=1$ is
\begin{equation}
    I_{\bw}^{l,1}(\bu)=-2\pi u\int_0^{\infty}\dd{w}w^{l+3}\exp\pqty{-Aw^2}
    \int_0^{\pi}\dd{\pqty{\cos\theta}}\pqty{\cos\theta}\exp\pqty{Rwu\cdot\cos\theta},
\end{equation}
where the inner angular integral is solved to give us
\begin{equation}
    I_R^{1}(\bu,\bw)=-\int_0^{\pi}\dd{\pqty{\cos\theta}}\pqty{\cos\theta}
    \exp\pqty{Rwu\cdot\cos\theta}=\frac{1}{Rwu}\pqty{e^{Rwu}+e^{-Rwu}}-
    \frac{1}{R^2w^2u^2}\pqty{e^{Rwu}-e^{-Rwu}}.
\end{equation}
Now, the center of mass velocity integral reads
\begin{equation}
    I_{\bw}^{1}(\bu)=\frac{2\pi}{R}\int_0^{\infty}\dd{w}w^{l+2}\exp\pqty{-Aw^2}
    \pqty{e^{Ru\cdot w}+e^{-Ru\cdot w}}
    -\frac{2\pi}{R^2u}\int_0^{\infty}\dd{w}w^{l+1}\exp\pqty{-Aw^2}
    \pqty{e^{Ru\cdot w}-e^{Ru\cdot w}}.
\end{equation}


\subsubsection{Shifted Gaussian integrals}
To proceed further, let us analyze the specific types of shifted Gaussian integrals,
namely
\begin{equation}
    I_{G,\pm}^n=\int_{0}^{\infty}\dd{x}x^n\exp\pqty{-ax^2\pm bx},
    \qquad n\in\Bqty{0,1,2,\dots}.
\end{equation}
To get a general solution for these types of integrals, let us write them in a
more canonical form first. To do so, let us introduce a variable transformation
\begin{equation}
    t:=\sqrt{a}x-\lambda,\qquad\lambda=\pm\frac{b}{2\sqrt{a}},
\end{equation}
this implies
\begin{equation}
    \begin{split}
        -ax^2\pm bx&=-t^2+\lambda^2,\\
        \dd{x}&=\frac{\dd{t}}{\sqrt{a}},\\
        x^n&=a^{-n/2}\cdot\pqty{t+\lambda}^n=a^{-n/2}\cdot
        \sum_{k=0}^{n}\pmqty{n\\k}\lambda^{n-k}t^k,\\
        \pmqty{n\\k}&=\frac{n!}{k!(n-k)!}.
    \end{split}
\end{equation}
Now, our shifted Gaussian integrals become
\begin{equation}
    I_{G,\pm}^{n}=\frac{\exp\pqty{\lambda^2}}{\sqrt{a^{n+1}}}\sum_{k=0}^{n}\pmqty{n\\k}
    \lambda^{n-k}\int_{-\lambda}^{\infty}\dd{t}t^ke^{-t^2}.
\end{equation}
The $\pm$ sign is now hidden in the parameter $\lambda$. Let us concentrate on the
canonical Gaussian integral
\begin{equation}
    G_k=\int_{-\lambda}^{\infty}\dd{t}t^ke^{-t^2}.
\end{equation}
Let us start with integration by parts and put $u=e^{-t^2}$ and $\dd{v}=t^k\dd{t}$.
This gives us $\dd{u}=-2te^{-t^2}\dd{t}$, and $v=t^{k+1}/(k+1)$. Now we have
\begin{equation}
    G_k=\frac{t^{k+1}}{k+1}\eval{e^{-t^2}}_{-\lambda}^{\infty}+
    \frac{2}{k+1}\int_{-\lambda}^{\infty}\dd{t}t^{k+2}e^{-t^2}=
    -(-1)^{k+1}\frac{\lambda^{k+1}}{k+1}e^{-\lambda^2}+\frac{2}{k+1}G_{k+2},
\end{equation}
a recurrent relation for the integral
\begin{equation}
    G_{k+2}=\frac{k+1}{2}G_k+(-\lambda)^{k+1}\cdot\frac{1}{2}e^{-\lambda^2}.
\end{equation}
In order to get a full solution, let us calculate the first two cases $k=0$ and $k=1$.
We have
\begin{equation}
    \begin{split}
        G_0&=\int_{-\lambda}^{\infty}\dd{t}e^{-t^2}=
        \frac{\sqrt{\pi}}{2}\pqty{1+\erf(\lambda)},\\
        G_1&=\int_{-\lambda}^{\infty}\dd{t}te^{-t^2}=\frac{1}{2}e^{-\lambda^2}.
    \end{split}
\end{equation}
Given this two functions, we can obtain the solution for any order $k$
\begin{equation}
    G_k=\frac{k-1}{2}G_{k-2}+(-\lambda)^{k-1}G_1,
\end{equation}
where we have rewritten our recurrent relation with the help of $G_1$.
Extending the the recurrent relation, we obtain two different results, for odd and even $k$.
For an even $k=2p$, then we have
\begin{equation}
    G_{2p}=\frac{\pqty{2p-1}!!}{2^p}G_0+G_1\sum_{j=1}^{p}
    \frac{1}{2^{j-1}}\frac{\pqty{2p-1}!!}{\pqty{2p-2j+1}!!}(-\lambda)^{2p-2j+1}.
\end{equation}
For odd valued $k=2p+1$, we have
\begin{equation}
    G_{2p+1}=G_1\sum_{j=1}^{p+1}\frac{1}{2^{j-1}}
    \frac{\pqty{2p}!!}{\pqty{2p-2j+2}!!}(-\lambda)^{2p-2j+2}.
\end{equation}

Now, we can write the solution of the original shifted Gaussian integrals as
\begin{equation}
    I_{G,\pm}^{n}(t)=\frac{\exp\pqty{\lambda^2}}{\sqrt{a^{n+1}}}\sum_{k=0}^{n}\pmqty{n\\k}
    \lambda^{n-k}G_{k}.
\end{equation}

\subsubsection{Symmetric Gaussian integrals}



\end{document}